% 若编译失败,且生成 .synctex(busy) 辅助文件,可能有两个原因:
% 1. 需要插入的图片不存在:Ctrl + F 搜索 'figure' 将这些代码注释/删除掉即可
% 2. 路径/文件名含中文或空格:更改路径/文件名即可

% ------------------------------------------------------------- %
% >> ------------------ 文章宏包及相关设置 ------------------ << %
% 设定文章类型与编码格式
\documentclass[UTF8]{report}		

% 本文特殊宏包
\usepackage{siunitx} % 埃米单位

% 本 .tex 专属的宏定义
    \def\V{\ \mathrm{V}}
    \def\mV{\ \mathrm{mV}}
    \def\kV{\ \mathrm{KV}}
    \def\KV{\ \mathrm{KV}}
    \def\MV{\ \mathrm{MV}}
    \def\A{\ \mathrm{A}}
    \def\mA{\ \mathrm{mA}}
    \def\kA{\ \mathrm{KA}}
    \def\KA{\ \mathrm{KA}}
    \def\MA{\ \mathrm{MA}}
    \def\O{\ \Omega}
    \def\mO{\ \Omega}
    \def\kO{\ \mathrm{K}\Omega}
    \def\KO{\ \mathrm{K}\Omega}
    \def\MO{\ \mathrm{M}\Omega}
    \def\Hz{\ \mathrm{Hz}}

% 自定义宏定义
    \def\N{\mathbb{N}}
    \def\F{\mathbb{F}}
    \def\Z{\mathbb{Z}}
    \def\Q{\mathbb{Q}}
    \def\R{\mathbb{R}}
    \def\C{\mathbb{C}}
    \def\T{\mathbb{T}}
    \def\S{\mathbb{S}}
    \def\A{\mathbb{A}}
    \def\I{\mathscr{I}}
    \def\Im{\mathrm{Im\,}}
    \def\Re{\mathrm{Re\,}}
    \def\d{\mathrm{d}}
    \def\p{\partial}

% 导入基本宏包
    \usepackage[UTF8]{ctex}     % 设置文档为中文语言
    \usepackage[colorlinks, linkcolor=blue, anchorcolor=blue, citecolor=blue, urlcolor=blue]{hyperref}  % 宏包:自动生成超链接 (此宏包与标题中的数学环境冲突)
    % \usepackage{hyperref}  % 宏包:自动生成超链接 (此宏包与标题中的数学环境冲突)
    % \hypersetup{
    %     colorlinks=true,    % false:边框链接 ; true:彩色链接
    %     citecolor={blue},    % 文献引用颜色
    %     linkcolor={blue},   % 目录 (我们在目录处单独设置),公式,图表,脚注等内部链接颜色
    %     urlcolor={orange},    % 网页 URL 链接颜色,包括 \href 中的 text
    %     % cyan 浅蓝色 
    %     % magenta 洋红色
    %     % yellow 黄色
    %     % black 黑色
    %     % white 白色
    %     % red 红色
    %     % green 绿色
    %     % blue 蓝色
    %     % gray 灰色
    %     % darkgray 深灰色
    %     % lightgray 浅灰色
    %     % brown 棕色
    %     % lime 石灰色
    %     % olive 橄榄色
    %     % orange 橙色
    %     % pink 粉红色
    %     % purple 紫色
    %     % teal 蓝绿色
    %     % violet 紫罗兰色
    % }

    % \usepackage{docmute}    % 宏包:子文件导入时自动去除导言区,用于主/子文件的写作方式,\include{./51单片机笔记}即可。注:启用此宏包会导致.tex文件capacity受限。
    \usepackage{amsmath}    % 宏包:数学公式
    \usepackage{mathrsfs}   % 宏包:提供更多数学符号
    \usepackage{amssymb}    % 宏包:提供更多数学符号
    \usepackage{pifont}     % 宏包:提供了特殊符号和字体
    \usepackage{extarrows}  % 宏包:更多箭头符号
    \usepackage{multicol}   % 宏包:支持多栏 
    \usepackage{graphicx}   % 宏包:插入图片
    \usepackage{float}      % 宏包:设置图片浮动位置
    %\usepackage{article}    % 宏包:使文本排版更加优美
    \usepackage{tikz}       % 宏包:绘图工具
    %\usepackage{pgfplots}   % 宏包:绘图工具
    \usepackage{enumerate}  % 宏包:列表环境设置
    \usepackage{enumitem}   % 宏包:列表环境设置

% 文章页面margin设置
    \usepackage[a4paper]{geometry}
        \geometry{top=1in}
        \geometry{bottom=1in}
        \geometry{left=0.75in}
        \geometry{right=0.75in}   % 设置上下左右页边距
        \geometry{marginparwidth=1.75cm}    % 设置边注距离(注释、标记等)

% 定义 solution 环境
\usepackage{amsthm}
\newtheorem{solution}{Solution}
        \geometry{bottom=1in}
        \geometry{left=0.75in}
        \geometry{right=0.75in}   % 设置上下左右页边距
        \geometry{marginparwidth=1.75cm}    % 设置边注距离(注释、标记等)

% 配置数学环境
    \usepackage{amsthm} % 宏包:数学环境配置
    % theorem-line 环境自定义
        \newtheoremstyle{MyLineTheoremStyle}% <name>
            {11pt}% <space above>
            {11pt}% <space below>
            {}% <body font> 使用默认正文字体
            {}% <indent amount>
            {\bfseries}% <theorem head font> 设置标题项为加粗
            {:}% <punctuation after theorem head>
            {.5em}% <space after theorem head>
            {\textbf{#1}\thmnumber{#2}\ \ (\,\textbf{#3}\,)}% 设置标题内容顺序
        \theoremstyle{MyLineTheoremStyle} % 应用自定义的定理样式
        \newtheorem{LineTheorem}{Theorem.\,}
    % theorem-block 环境自定义
        \newtheoremstyle{MyBlockTheoremStyle}% <name>
            {11pt}% <space above>
            {11pt}% <space below>
            {}% <body font> 使用默认正文字体
            {}% <indent amount>
            {\bfseries}% <theorem head font> 设置标题项为加粗
            {:\\ \indent}% <punctuation after theorem head>
            {.5em}% <space after theorem head>
            {\textbf{#1}\thmnumber{#2}\ \ (\,\textbf{#3}\,)}% 设置标题内容顺序
        \theoremstyle{MyBlockTheoremStyle} % 应用自定义的定理样式
        \newtheorem{BlockTheorem}[LineTheorem]{Theorem.\,} % 使用 LineTheorem 的计数器
    % definition 环境自定义
        \newtheoremstyle{MySubsubsectionStyle}% <name>
            {11pt}% <space above>
            {11pt}% <space below>
            {}% <body font> 使用默认正文字体
            {}% <indent amount>
            {\bfseries}% <theorem head font> 设置标题项为加粗
           % {:\\ \indent}% <punctuation after theorem head>
            {\\\indent}
            {0pt}% <space after theorem head>
            {\textbf{#3}}% 设置标题内容顺序
        \theoremstyle{MySubsubsectionStyle} % 应用自定义的定理样式
        \newtheorem{definition}{}

%宏包:有色文本框(proof环境)及其设置
    \usepackage[dvipsnames,svgnames]{xcolor}    %设置插入的文本框颜色
    \usepackage[strict]{changepage}     % 提供一个 adjustwidth 环境
    \usepackage{framed}     % 实现方框效果
        \definecolor{graybox_color}{rgb}{0.95,0.95,0.96} % 文本框颜色。修改此行中的 rgb 数值即可改变方框纹颜色,具体颜色的rgb数值可以在网站https://colordrop.io/ 中获得。(截止目前的尝试还没有成功过,感觉单位不一样)(找到喜欢的颜色,点击下方的小眼睛,找到rgb值,复制修改即可)
        \newenvironment{graybox}{%
        \def\FrameCommand{%
        \hspace{1pt}%
        {\color{gray}\small \vrule width 2pt}%
        {\color{graybox_color}\vrule width 4pt}%
        \colorbox{graybox_color}%
        }%
        \MakeFramed{\advance\hsize-\width\FrameRestore}%
        \noindent\hspace{-4.55pt}% disable indenting first paragraph
        \begin{adjustwidth}{}{7pt}%
        \vspace{2pt}\vspace{2pt}%
        }
        {%
        \vspace{2pt}\end{adjustwidth}\endMakeFramed%
        }



% 外源代码插入设置
    % matlab 代码插入设置
    \usepackage{matlab-prettifier}
        \lstset{style=Matlab-editor}    % 继承 matlab 代码高亮 , 此行不能删去
    \usepackage[most]{tcolorbox} % 引入tcolorbox包 
    \usepackage{listings} % 引入listings包
        \tcbuselibrary{listings, skins, breakable}
        \newfontfamily\codefont{Consolas} % 定义需要的 codefont 字体
        \lstdefinestyle{MatlabStyle_inc}{   % 插入代码的样式
            language=Matlab,
            basicstyle=\small\ttfamily\codefont,    % ttfamily 确保等宽 
            breakatwhitespace=false,
            breaklines=true,
            captionpos=b,
            keepspaces=true,
            numbers=left,
            numbersep=15pt,
            showspaces=false,
            showstringspaces=false,
            showtabs=false,
            tabsize=2,
            xleftmargin=15pt,   % 左边距
            %frame=single, % single 为包围式单线框
            frame=shadowbox,    % shadowbox 为带阴影包围式单线框效果
            %escapeinside=``,   % 允许在代码块中使用 LaTeX 命令 (此行无用)
            %frameround=tttt,    % tttt 表示四个角都是圆角
            framextopmargin=0pt,    % 边框上边距
            framexbottommargin=0pt, % 边框下边距
            framexleftmargin=5pt,   % 边框左边距
            framexrightmargin=5pt,  % 边框右边距
            rulesepcolor=\color{red!20!green!20!blue!20}, % 阴影框颜色设置
            %backgroundcolor=\color{blue!10}, % 背景颜色
        }
        \lstdefinestyle{MatlabStyle_src}{   % 插入代码的样式
            language=Matlab,
            basicstyle=\small\ttfamily\codefont,    % ttfamily 确保等宽 
            breakatwhitespace=false,
            breaklines=true,
            captionpos=b,
            keepspaces=true,
            numbers=left,
            numbersep=15pt,
            showspaces=false,
            showstringspaces=false,
            showtabs=false,
            tabsize=2,
        }
        \newtcblisting{matlablisting}{
            %arc=2pt,        % 圆角半径
            % 调整代码在 listing 中的位置以和引入文件时的格式相同
            top=0pt,
            bottom=0pt,
            left=-5pt,
            right=-5pt,
            listing only,   % 此句不能删去
            listing style=MatlabStyle_src,
            breakable,
            colback=white,   % 选一个合适的颜色
            colframe=black!0,   % 感叹号后跟不透明度 (为 0 时完全透明)
        }
        \lstset{
            style=MatlabStyle_inc,
        }



% table 支持
    \usepackage{booktabs}   % 宏包:三线表
    %\usepackage{tabularray} % 宏包:表格排版
    %\usepackage{longtable}  % 宏包:长表格
    %\usepackage[longtable]{multirow} % 宏包:multi 行列


% figure 设置
\usepackage{graphicx}   % 支持 jpg, png, eps, pdf 图片 
\usepackage{float}      % 支持 H 选项
\usepackage{svg}        % 支持 svg 图片
\usepackage{subcaption} % 支持子图
\svgsetup{
        % 指向 inkscape.exe 的路径
       inkscapeexe = C:/aa_MySame/inkscape/bin/inkscape.exe, 
        % 一定程度上修复导入后图片文字溢出几何图形的问题
       inkscapelatex = false                 
   }

% 图表进阶设置
    \usepackage{caption}    % 图注、表注
        \captionsetup[figure]{name=图}  
        \captionsetup[table]{name=表}
        \captionsetup{
            labelfont=bf, % 设置标签为粗体
            textfont=bf,  % 设置文本为粗体
            font=small  
        }
    \usepackage{float}     % 图表位置浮动设置 
        % \floatstyle{plaintop} % 设置表格标题在表格上方
        % \restylefloat{table}  % 应用设置


% 圆圈序号自定义
    \newcommand*\circled[1]{\tikz[baseline=(char.base)]{\node[shape=circle,draw,inner sep=0.8pt, line width = 0.03em] (char) {\small \bfseries #1};}}   % TikZ solution


% 列表设置
    \usepackage{enumitem}   % 宏包:列表环境设置
        \setlist[enumerate]{
            label=\bfseries(\arabic*) ,   % 设置序号样式为加粗的 (1) (2) (3)
            ref=\arabic*, % 如果需要引用列表项,这将决定引用格式(这里仍然使用数字)
            itemsep=0pt, parsep=0pt, topsep=0pt, partopsep=0pt, leftmargin=3.5em} 
        \setlist[itemize]{itemsep=0pt, parsep=0pt, topsep=0pt, partopsep=0pt, leftmargin=3.5em}
        \newlist{circledenum}{enumerate}{1} % 创建一个新的枚举环境  
        \setlist[circledenum,1]{  
            label=\protect\circled{\arabic*}, % 使用 \arabic* 来获取当前枚举计数器的值,并用 \circled 包装它  
            ref=\arabic*, % 如果需要引用列表项,这将决定引用格式(这里仍然使用数字)
            itemsep=0pt, parsep=0pt, topsep=0pt, partopsep=0pt, leftmargin=3.5em
        }  

% 文章默认字体设置
    \usepackage{fontspec}   % 宏包:字体设置
        \setmainfont{STKaiti}    % 设置中文字体为宋体字体
        \setCJKmainfont[AutoFakeBold=3]{STKaiti} % 设置加粗字体为 STKaiti 族,AutoFakeBold 可以调整字体粗细
        \setmainfont{Times New Roman} % 设置英文字体为Times New Roman


% 其它设置
    % 脚注设置
    \renewcommand\thefootnote{\ding{\numexpr171+\value{footnote}}}
    % 参考文献引用设置
        \bibliographystyle{unsrt}   % 设置参考文献引用格式为unsrt
        \newcommand{\upcite}[1]{\textsuperscript{\cite{#1}}}     % 自定义上角标式引用
    % 文章序言设置
        \newcommand{\cnabstractname}{序言}
        \newenvironment{cnabstract}{%
            \par\Large
            \noindent\mbox{}\hfill{\bfseries \cnabstractname}\hfill\mbox{}\par
            \vskip 2.5ex
            }{\par\vskip 2.5ex}


% 各级标题自定义设置
    \usepackage{titlesec}   
    % chapter
        \titleformat{\chapter}[hang]{\normalfont\Large\bfseries\centering}{Homework \thechapter }{10pt}{}
        \titlespacing*{\chapter}{0pt}{-30pt}{10pt} % 控制上方空白的大小
    % section
        \titleformat{\section}[hang]{\normalfont\large\bfseries}{\thesection}{8pt}{}
    % subsection
        %\titleformat{\subsubsection}[hang]{\normalfont\bfseries}{}{8pt}{}
    % subsubsection
        %\titleformat{\subsubsection}[hang]{\normalfont\bfseries}{}{8pt}{}


% >> ------------------ 文章宏包及相关设置 ------------------ << %
% ------------------------------------------------------------- %



% ----------------------------------------------------------- %
% >> --------------------- 文章信息区 --------------------- << %
% 页眉页脚设置

\usepackage{fancyhdr}   %宏包:页眉页脚设置
    \pagestyle{fancy}
    \fancyhf{}
    \cfoot{\thepage}
    \renewcommand\headrulewidth{1pt}
    \renewcommand\footrulewidth{0pt}
    \chead{信号与系统作业,\ 尹超,\ 2023K8009926003}
    \lhead{Homework}
    \rhead{yinchao23@mails.ucas.ac.cn}

%文档信息设置
\title{信号与系统作业\\ Homework}
\author{尹超\\ \footnotesize 中国科学院大学,北京 100049\\ Carter Yin \\ \footnotesize University of Chinese Academy of Sciences, Beijing 100049, China}
\date{\footnotesize 2024.8 -- 2025.1}
% >> --------------------- 文章信息区 --------------------- << %
% ----------------------------------------------------------- %     


% 开始编辑文章

\begin{document}
\zihao{5}           % 设置全文字号大小

% --------------------------------------------------------------- %
% >> --------------------- 封面序言与目录 --------------------- << %
% 封面
    \maketitle\newpage  
    \pagenumbering{Roman} % 页码为大写罗马数字
    \thispagestyle{fancy}   % 显示页码、页眉等

% 序言
    \begin{cnabstract}\normalsize 
        本文为笔者概率论与数理统计的作业。\par
        望老师批评指正。
    \end{cnabstract}
    \addcontentsline{toc}{chapter}{序言} % 手动添加为目录

% % 不换页目录
%     \setcounter{tocdepth}{0}
%     \noindent\rule{\textwidth}{0.1em}   % 分割线
%     \noindent\begin{minipage}{\textwidth}\centering 
%         \vspace{1cm}
%         \tableofcontents\thispagestyle{fancy}   % 显示页码、页眉等   
%     \end{minipage}  
%     \addcontentsline{toc}{chapter}{目录} % 手动添加为目录

% 目录
\setcounter{tocdepth}{4}                % 目录深度(为1时显示到section)
\tableofcontents                        % 目录页
\addcontentsline{toc}{chapter}{目录}    % 手动添加此页为目录
\thispagestyle{fancy}                   % 显示页码、页眉等 

% 收尾工作
    \newpage    
    \pagenumbering{arabic} 

% >> --------------------- 封面序言与目录 --------------------- << %
% --------------------------------------------------------------- %


\chapter{电路模型和电路定律}

\section{电路和电路模型}

\begin{definition}
    \textbf{5种基本的理想电路元件}
    \begin{itemize}
        \item 电阻元件:表示消耗电能的元件
        \item 电感元件:表示产生磁场,储存磁场能量的元件
        \item 电容元件:表示产生电场,储存电场能量的元件
        \item 电压源和电流源:表示将其它形式的能量转变成电能的元件
    \end{itemize}
    \textbf{注意}
    \begin{itemize}
        \item 5种基本理想电路元件有三个特征:
        \begin{enumerate}[label=(\alph*)]
            \item 只有两个端子
            \item 可以用电压、电流按数学方式描述
            \item 不能被分解为其他元件
        \end{enumerate}
    \end{itemize}
\end{definition}

\chapter{第一章作业}\thispagestyle{fancy}

\section{习题总结}

\subsection{反证法}

\begin{definition}
    \textbf{间接证明法(indirect proof)}\par
    直接证明法有的时候比较困难\par
    不从前提开始、以结论结束的证明方法叫间接证明法\par
    \vspace{1em} % 添加一个空行 
    反证法(proof by contraposition)\par
    归谬证明法(proof by contradiction)\par
\vspace{1em} % 添加一个空行
    \textbf{反证法(proof by contraposition)}\par
    条件语句 $p \rightarrow q$ 等价于它的逆否命题 $\neg q \rightarrow \neg p$\par
    证明当 $\neg q$ 为真时,$\neg p$ 一定为真\par
\vspace{1em} % 添加一个空行
    \textbf{示例1:}\par
    证明“如果 $n$ 是一个整数且 $3n + 2$ 是奇数,则 $n$ 是奇数”\par
    直接证明比较困难\par
    假设 $n$ 不是奇数,即 $n$ 为偶数,则 $n = 2k$,$k$ 为某个整数\par
    $3n + 2 = 3(2k) + 2 = 6k + 2 = 2(3k + 1)$,即 $3n + 2$ 为偶数,逆否命题为真,所以原命题也为真\par
\vspace{1em} % 添加一个空行
    \textbf{示例2:}\par
    证明“如果 $n$ 是一个整数且 $n^2$ 是奇数,则 $n$ 是奇数”\par
    直接证明比较困难:假设 $n^2$ 是奇数,很难推导下去\par
    假设 $n$ 不是奇数,即 $n$ 为偶数,则 $n = 2k$,$k$ 为某个整数\par
    $n^2 = (2k)^2 = 4k^2 = 2(2k^2)$,即 $n^2$ 为偶数\par
    即我们证明了逆否命题“如果 $n$ 是一个偶数,则 $n^2$ 是偶数”\par
    由反证法,“如果 $n$ 是一个整数且 $n^2$ 是奇数,则 $n$ 是奇数”\par
\end{definition}


\textbf{示例:}\par
        \[
        A =
        \begin{bmatrix}
        1 & 0 & 1 \\
        0 & 1 & 0
        \end{bmatrix}
        , \quad
        B =
        \begin{bmatrix}
        0 & 1 & 0 \\
        1 & 1 & 0
        \end{bmatrix}
        \]
        \[
        A \lor B =
        \begin{bmatrix}
        1 \lor 0 & 0 \lor 1 & 1 \lor 0 \\
        0 \lor 1 & 1 \lor 1 & 0 \lor 0
        \end{bmatrix}
        =
        \begin{bmatrix}
        1 & 1 & 1 \\
        1 & 1 & 0
        \end{bmatrix}
        \]
        \[
        A \land B =
        \begin{bmatrix}
        1 \land 0 & 0 \land 1 & 1 \land 0 \\
        0 \land 1 & 1 \land 1 & 0 \land 0
        \end{bmatrix}
        =
        \begin{bmatrix}
        0 & 0 & 0 \\
        0 & 1 & 0
        \end{bmatrix}
        \]
    

        \chapter{基础:逻辑和证明}\thispagestyle{fancy} 

        \section{逻辑}
        
        
        
        \subsection{真值表}
        \begin{definition}
        下面是一个包含所有真值运算的真值表(包括异或运算)
        \begin{table}[H]
        \centering
        \begin{tabular}{|c|c|c|c|c|c|c|c|c|}
        \hline
        $p$ & $q$ & $\neg p$ & $p \land q$ & $p \lor q$ & $p \rightarrow q$ & $p \leftrightarrow q$ & $p \oplus q$  \\
        \hline
        T & T & F & T & T & T & T & F  \\
        T & F & F & F & T & F & F & T  \\
        F & T & T & F & T & T & F & T  \\
        F & F & T & F & F & T & T & F  \\
        \hline
        \end{tabular}
        \end{table}
        \end{definition}
        
        \subsection{比特运算真值表}
        \begin{definition}
        下面是一个和上面的表对应的比特运算真值表
        \begin{table}[H]
        \centering
        \begin{tabular}{|c|c|c|c|c|c|c|c|}
        \hline
        $p$ & $q$ & $\neg p$ & $p \land q$ & $p \lor q$ & $p \rightarrow q$ & $p \leftrightarrow q$ & $p \oplus q$  \\
        \hline
        1 & 1 & 0 & 1 & 1 & 1 & 1 & 0  \\
        1 & 0 & 0 & 0 & 1 & 0 & 0 & 1  \\
        0 & 1 & 1 & 0 & 1 & 1 & 0 & 1  \\
        0 & 0 & 1 & 0 & 0 & 1 & 1 & 0  \\
        \hline
        \end{tabular}
        \end{table}
        \end{definition}
        
        



        \chapter{事件的概率}\thispagestyle{fancy} 
        \section{Space}
        
        
        \begin{definition}[Probability Space]
        • 概率模型的三个要素, (\(\Omega\), \(\Sigma\), \(P\))\par
        • Samples space, event sets, probability measure\par
        • \(\Sigma\): the set of subsets
        \end{definition}
        
        




\begin{equation}\label{公式1}
\frac{u_{j}^{k}-u_{j}^{k-1}}{h_t}=a\theta\frac{u_{j+1}^{k}-2u_{j}^{k}+u_{j-1}^{k}}{h_x^2}+a(1-\theta)\frac{u_{j+1}^{k-1}-2u_{j}^{k-1}+u_{j-1}^{k-1}}{h_x^2}
\end{equation}

其中 $\theta \in [0, 1]$ 为权重,其截断误差 $R = a\left(\frac{1}{2}-\theta\right)h_t\left[\frac{\partial^{3}u}{\partial x^{2}\partial t}\right]_{j}^{k}+O(h_t^{2}+h_x^2)$,因此当 $\theta = \frac{1}{2}$ 时,方程具有 $O(h_t^{2}+h_x^2)$ 精度,称为 Crank-Nicolson 格式(CN 格式)。


公式 \ref{公式5} 的增长因子及稳定性条件为:

\begin{equation}
   G(h_t,\sigma)=\frac{1-4(1-\theta)ar\sin^2\frac{\sigma h}2}{1+4\theta ar\sin^2\frac{\sigma h}2}, \ \ 
   \begin{cases}
       r\leqslant\frac{1}{2a(1-2\theta)}, & \theta \in [0, \frac{1}{2}) \\ 
       \text{无条件稳定}, & \theta \in [\frac{1}{2}, 1] \\ 
   \end{cases}
\end{equation}


\begin{LineTheorem}[这是一个 Line Theorem]\label{这是一个 Line Theorem}
   你好你好你好
\end{LineTheorem}

\begin{BlockTheorem}[这是一个 Block Theorem]\label{这是一个 Block Theorem}
   你好你好你好
\end{BlockTheorem}



\begin{graybox}
\textbf{定理 \ref{这是一个 Block Theorem} 的证明:}\\
你好你好你好
\end{graybox}



\cleardoublepage
\section{附录:拟合用Python 代码}

\begin{lstlisting}[language=Python, caption=Figure's Python code, label=code:python_example]
#Figure_1.py
    import numpy as np
import matplotlib.pyplot as plt
from matplotlib import rcParams

# 设置中文字体
rcParams['font.sans-serif'] = ['SimHei']  # 使用黑体
rcParams['axes.unicode_minus'] = False  # 解决负号显示问题

# 数据
n = np.array([0, 1, 2, 3, 4])
T_ms = np.array([1586.011, 1672.654, 1755.244, 1833.753, 1909.52])
T_s = T_ms / 1000  # 将时间单位从毫秒转换为秒
m1 = (213.03+2.60) / 1000  # 滑块加条形挡光片质量,单位为kg
m2 = 25.02 / 1000    # 骑码质量,单位为kg
m = m1 + n * m2  # 总质量,单位为kg

# 进行线性拟合
coefficients = np.polyfit(m, T_s**2, 1)
poly = np.poly1d(coefficients)

# 计算相关系数
correlation_matrix = np.corrcoef(m, T_s**2)
correlation_coefficient = correlation_matrix[0, 1]

# 拟合结果
print(f"拟合得到的系数: {coefficients}")
print(f"相关系数: {correlation_coefficient}")

# 绘图
plt.scatter(m, T_s**2, label='实验数据')
plt.plot(m, poly(m), label='拟合曲线', color='red')

# 显示图线方程和相关系数
equation_text = f'$T^2 = {coefficients[0]:.4f} \cdot m + {coefficients[1]:.4f}$\n相关系数: {correlation_coefficient:.4f}'
plt.text(0.45, 0.45, equation_text, transform=plt.gca().transAxes, fontsize=12, verticalalignment='top')

plt.xlabel('质量 $m$ (kg)')
plt.ylabel('周期平方 $T^2$ (s$^2$)')
plt.legend()
plt.title('$T^2 - m$ 拟合')
plt.show()

\end{lstlisting}






%\begin{figure}[H]
%    \centering
%    \includegraphics[width=0.5\textwidth]{assets/差分格式示意图.pdf}
%    \caption{\textbf{插入pdf图片}}\label{插入pdf图片}
%\end{figure}




%表格三线表:

\begin{table}[H]
   \centering
   \caption{\textbf{符号含义与约定}}
   \label{tab:waterpump}
   \begin{tabular}{ccccc}
   \toprule
   符号 & 符号含义& 单位\\
   \midrule
   符号1& 含义1& 单位1\\
   符号2& 含义2& 单位2\\
   符号3& 含义3& 单位3\\
   符号4& 含义4& 单位4\\
   \bottomrule
   \end{tabular}
\end{table}






















% ----------------------------------------------------------- %
% >> ---------------------- 参考文献 ---------------------- << %
\nocite{*}
\bibliography{re}
\thispagestyle{fancy} 
\addcontentsline{toc}{chapter}{参考文献}
%这里要用到 bibtex,使用xelatex->bibtex->xelatex->xelatex编译链
%同时要把re.bib文件放在同一目录下
%下面是re.bib文件的内容
% @book{knuth1984texbook,
%   author    = {Donald E. Knuth},
%   title     = {The TeXbook},
%   year      = {1984},
%   publisher = {Addison-Wesley},
% }

% @article{lamport1994latex,
%   author  = {Leslie Lamport},
%   title   = {LaTeX: A Document Preparation System},
%   journal = {Addison-Wesley},
%   year    = {1994},
% }

% @inproceedings{goossens1993latex,
%   author    = {Michel Goossens and Frank Mittelbach and Alexander Samarin},
%   title     = {The LaTeX Companion},
%   booktitle = {Addison-Wesley Series on Tools and Techniques for Computer Typesetting},
%   year      = {1993},
% }

% @misc{wikibibtex,
%   author       = {Wikipedia contributors},
%   title        = {BibTeX --- Wikipedia{,} The Free Encyclopedia},
%   year         = {2024},
%   url          = {https://en.wikipedia.org/wiki/BibTeX},
%   note         = {Accessed: 2025-04-15}
% }



% >> ---------------------- 参考文献 ---------------------- << %
% ----------------------------------------------------------- %



% ------------------------------------------------------------ %
% >> ------------------------ 附录 ------------------------ << %

% 附录设置
\newpage
\appendix
% chapter 标题自定义设置
\titleformat{\chapter}[hang]{\normalfont\huge\bfseries\centering}{}{20pt}{}
\titlespacing*{\chapter}{0pt}{-25pt}{8pt} % 控制上方空白的大小
% section 标题自定义设置 
\titleformat{\section}[hang]{\normalfont\centering\Large\bfseries}{\thesection}{8pt}{}




% 附录 A
\chapter*{附录 A. 中英文对照表}\addcontentsline{toc}{chapter}{附录 A. 中英文对照表}   
\thispagestyle{fancy} 
\setcounter{section}{0}   
\renewcommand\thesection{A.\arabic{section}}   
\renewcommand{\thefigure}{A.\arabic{figure}} 
\renewcommand{\thetable}{A.\arabic{table}}

\section{中英文对照表}
\begin{multicols}{2}  

   \begin{table}[H]\centering
   \caption{\textbf{中英文对照表}}
   \begin{tabular}{cccccccc}\toprule
       English & 中文 \\
       \midrule
       voltage            & 电压 \\
       current            & 电流 \\
       power              & 功率 \\
       resistance         & 电阻 \\
       conductance        & 电导 \\
       inductance         & 电感 \\
       capacitance        & 电容 \\
       frequency          & 频率 \\
       circuit            & 电路 \\
       circuit element    & 电流元件 \\
       signal             & 信号 \\
       circuit analysis   & 电路分析 \\
       circuit synthesis  & 电路综合 \\
       circuit design     & 电路设计 \\
       circuit topology   & 电路拓扑 \\
       \bottomrule
   \end{tabular}
   \end{table}
    
   \begin{table}[H]\centering
       \caption{\textbf{中英文对照表}}
       \begin{tabular}{cccccccc}\toprule
           English & 中文 \\
           \midrule
           voltage            & 电压 \\
           current            & 电流 \\
           power              & 功率 \\
           resistance         & 电阻 \\
           conductance        & 电导 \\
           inductance         & 电感 \\
           capacitance        & 电容 \\
           frequency          & 频率 \\
           circuit            & 电路 \\
           circuit element    & 电流元件 \\
           signal             & 信号 \\
           circuit analysis   & 电路分析 \\
           circuit synthesis  & 电路综合 \\
           circuit design     & 电路设计 \\
           circuit topology   & 电路拓扑 \\
           \bottomrule
       \end{tabular}
   \end{table}
\end{multicols} 
    
\section{支撑材料列表} 

\begin{center}
 这里插入一张图片(类似思维导图那种)
\end{center}


% 附录 B
\chapter*{附录 B. 代码}\addcontentsline{toc}{chapter}{附录 B. 代码}   
\thispagestyle{fancy} 
\setcounter{section}{0}   
\renewcommand\thesection{B.\arabic{section}}   
\renewcommand{\thefigure}{B.\arabic{figure}} 
\renewcommand{\thetable}{B.\arabic{table}}

% 注意:listing环境中手动输入的代码需要顶格写

\begin{matlablisting}
MATLAB code here
x = 0:0.1:2*pi;
y = sin(x);
plot(x, y);
xlabel('x');
ylabel('sin(x)');
title('Sine Function');
... (MATLAB code here,最好是插入文件)
MATLAB code here
x = 0:0.1:2*pi;
y = sin(x);
plot(x, y);
xlabel('x');
ylabel('sin(x)');
title('Sine Function');
... (MATLAB code here,最好是插入文件)
MATLAB code here
x = 0:0.1:2*pi;
y = sin(x);
plot(x, y);
xlabel('x');
ylabel('sin(x)');
title('Sine Function');
... (MATLAB code here,最好是插入文件)
MATLAB code here
x = 0:0.1:2*pi;
y = sin(x);
plot(x, y);
xlabel('x');
ylabel('sin(x)');
title('Sine Function');
... (MATLAB code here,最好是插入文件)
MATLAB code here
x = 0:0.1:2*pi;
y = sin(x);
plot(x, y);
xlabel('x');
ylabel('sin(x)');
title('Sine Function');
... (MATLAB code here,最好是插入文件)
MATLAB code here
x = 0:0.1:2*pi;
y = sin(x);
plot(x, y);
xlabel('x');
ylabel('sin(x)');
title('Sine Function');
... (MATLAB code here,最好是插入文件)% ... (MATLAB code here,最好是插入文件)% ... (MATLAB code here,最好是插入文件)% ... (MATLAB code here,最好是插入文件)% ... (MATLAB code here,最好是插入文件)A
MATLAB code here
x = 0:0.1:2*pi;
y = sin(x);
plot(x, y);
xlabel('x');
ylabel('sin(x)');
title('Sine Function');
... (MATLAB code here,最好是插入文件)
\end{matlablisting}


% >> ------------------------ 附录 ------------------------ << %
% ------------------------------------------------------------ %

\end{document}

% VScode 常用快捷键:

% Ctrl + R:                 打开最近的文件夹
% F2:                       变量重命名
% Ctrl + Enter:             行中换行
% Alt + up/down:            上下移行
% 鼠标中键 + 移动:           快速多光标
% Shift + Alt + up/down:    上下复制
% Ctrl + left/right:        左右跳单词
% Ctrl + Backspace/Delete:  左右删单词    
% Shift + Delete:           删除此行
% Ctrl + J:                 打开 VScode 下栏(输出栏)
% Ctrl + B:                 打开 VScode 左栏(目录栏)
% Ctrl + `:                 打开 VScode 终端栏
% Ctrl + 0:                 定位文件
% Ctrl + Tab:               切换已打开的文件(切标签)
% Ctrl + Shift + P:         打开全局命令(设置)

% Latex 常用快捷键

% Ctrl + Alt + J:           由代码定位到PDF
% 


% Git提交规范:
% update: Linear Algebra 2 notes
% add: Linear Algebra 2 notes
% import: Linear Algebra 2 notes
% delete: Linear Algebra 2 notes
